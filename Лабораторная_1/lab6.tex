\documentclass[a4paper,12pt]{article}

\usepackage{cmap}
\usepackage[T2A]{fontenc}
\usepackage[utf8x]{inputenc}
\usepackage[english, russian]{babel}

\usepackage{misccorr} % в заголовках появляется точка, но при ссылке на них ее нет
\usepackage{amssymb,amsfonts,amsmath,amsthm}  
\usepackage{indentfirst}
\usepackage[usenames,dvipsnames]{color} 
\usepackage[unicode, colorlinks, urlcolor=magenta, linkcolor=black, pagecolor=black]{hyperref}
\usepackage{makecell,multirow} 
\usepackage{ulem}
\usepackage{graphicx}
\graphicspath{{img/}}
\usepackage{geometry}
\geometry{left=1.5cm,right=1.5cm,top=3cm,bottom=3cm,bindingoffset=0cm}
\usepackage{fancyhdr} 
\linespread{1.3} 
\frenchspacing 
\renewcommand{\labelenumii}{\theenumii)} 
\usepackage{caption}
%%%%%%%%%%%%%%%%%%%%%%%%%%%%%%%%%%%%%%%%%%%%%%%%%%%%%%%%%%%%%%%%%%%%%%%%%%%%%%%
%%%%%%%%%%%%%%%%%%%%%%%%%%%%%%%%%%%%%%%%%%%%%%%%%%%%%%%%%%%%%%%%%%%%%%%%%%%%%%%

\def\labauthor{Сарафанов Ф.Г.}
\def\labauthors{Сарафанов Ф.Г.}
\def\labnumber{13}
\def\labtheme{Маятник Обербека}

%%%%%%%%%%%%%%%%%%%%%%%%%%%%%%%%%%%%%%%%%%%%%%%%%%%%%%%%%%%%%%%%%%%%%%%%%%%%%%%
%%%%%%%%%%%%%%%%%%%%%%%%%%%%%%%%%%%%%%%%%%%%%%%%%%%%%%%%%%%%%%%%%%%%%%%%%%%%%%%
	%применим колонтитул к стилю страницы
\pagestyle{fancy} 
	%очистим "шапку" страницы
\fancyhead{} 
	%слева сверху на четных и справа на нечетных
\fancyhead[LE,RO]{\labauthors} 
	%справа сверху на четных и слева на нечетных
\fancyhead[LO, RE]{Отчёт по лабораторной работе №\labnumber} 
	%очистим "подвал" страницы
\fancyfoot{} 
	% номер страницы в нижнем колинтуле в центре
\fancyfoot[CO,CE]{\thepage} 

%%%%%%%%%%%%%%%%%%%%%%%%%%%%%%%%%%%%%%%%%%%%%%%%%%%%%%%%%%%%%%%%%%%%%%%%%%%%%%% % колинтулы на страницах
%%%%%%%%%%%%%%%%%%%%%%%%%%%%%%%%%%%%%%%%%%%%%%%%%%%%%%%%%%%%%%%%%%%%%%%%%%%%%%%
\usepackage{float}
\usepackage[mode=buildnew]{standalone}
\usepackage{tikz} 
\usepackage{subcaption}
\usepackage{tikz,csvsimple}
\usetikzlibrary{scopes}
\usetikzlibrary{%
     decorations.pathreplacing,%
     decorations.pathmorphing,%
    patterns,%
    calc,%
    scopes,%
    arrows,%
    % arrows.spaced,%
}

\makeatletter
\newif\if@gather@prefix 
\preto\place@tag@gather{% 
  \if@gather@prefix\iftagsleft@ 
    \kern-\gdisplaywidth@ 
    \rlap{\gather@prefix}% 
    \kern\gdisplaywidth@ 
  \fi\fi 
} 
\appto\place@tag@gather{% 
  \if@gather@prefix\iftagsleft@\else 
    \kern-\displaywidth 
    \rlap{\gather@prefix}% 
    \kern\displaywidth 
  \fi\fi 
  \global\@gather@prefixfalse 
} 
\preto\place@tag{% 
  \if@gather@prefix\iftagsleft@ 
    \kern-\gdisplaywidth@ 
    \rlap{\gather@prefix}% 
    \kern\displaywidth@ 
  \fi\fi 
} 
\appto\place@tag{% 
  \if@gather@prefix\iftagsleft@\else 
    \kern-\displaywidth 
    \rlap{\gather@prefix}% 
    \kern\displaywidth 
  \fi\fi 
  \global\@gather@prefixfalse 
} 
\newcommand*{\beforetext}[1]{% 
  \ifmeasuring@\else
  \gdef\gather@prefix{#1}% 
  \global\@gather@prefixtrue 
  \fi
} 
\makeatother

\usepackage{booktabs}
\usepackage{pgfplots, pgfplotstable}

\pgfplotstableset{
    columns/mat/.style={%
    string type,
    column type=r,% 
    column name=\textsc{Груз}%
    },
    columns/m1/.style={column name=$m_1\text{, г}$},
    columns/m2/.style={column name=$m_2\text{, г}$},
    columns/m3/.style={column name=$m_3\text{, г}$},
    columns/m/.style={column name=$m\text{, г}$},
    columns/dr/.style={column name=$\Delta{d}\text{, мм}$},
    columns/d/.style={column name=$<d>$},
    columns/r/.style={column name=$<r>$},
    empty cells with={--}, % replace empty cells with ’--’
    every head row/.style={before row=\toprule,after row=\midrule},
    every last row/.style={after row=\bottomrule}
  }

\tikzset{force/.style={>=latex,draw=blue,fill=blue}, axis/.style={densely dashed,gray,font=\small}, acceleration/.style={>=open triangle 60,draw=blue,fill=blue}, inforce/.style={force,double equal sign distance=2pt}, interface/.style={pattern = north east lines, draw    = none, pattern color=gray!60, }, cross/.style={cross out, draw=black, minimum size=2*(#1-\pgflinewidth), inner sep=0pt, outer sep=0pt},    cargo/.style={rectangle, fill=black!70, inner sep=2.5mm/10, }}
\usepackage[outline]{contour}

% \begin{filecontents}{TableForErrorBarQuestion.txt}
% xValue2 yValue2 Deltay2 Deltax2
% 0   53366   2761 0.005
% 2.81   64784   2604 0.005
% 9.75   153559   5195 0.005
% 17.5   338088   11102 0.005 
% \end{filecontents}


\begin{document}

\begin{titlepage}

\begin{center}

{\small\textsc{Нижегородский государственный университет имени Н.\,И. Лобачевского}}
\vskip 1pt \hrule \vskip 3pt
{\small\textsc{Радиофизический факультет}}

\vfill

{\Large Отчет по лабораторной работе №\labnumber\vskip 12pt\bfseries \labtheme}
	
\end{center}

\vfill
	
\begin{flushright}
	{Выполнил студент 410 группы\\ \labauthor}%\vskip 12pt Принял:\\ Менсов С.\,Н.}
\end{flushright}
	
\vfill
	
\begin{center}
	Нижний Новгород, 2016
\end{center}

\end{titlepage}



\tableofcontents
\newpage

\section*{Введение} % (fold)
\addcontentsline{toc}{section}{Введение}
\label{sec:input}


\section{Оборудование} % (fold)
\label{sec:device}


\begin{figure}[H]
	\centering
	\includestandalone[width=0.8\linewidth]{img/device}
	\caption{Эскиз установки}
	\label{fig:device}
\end{figure}


Для проведения опытов использовались: установка, состоящая из крестовины -- 4 спиц, закрепленных резьбовым соединением на втулке под углом $90^\circ$ друг у другу, втулки (ось втулки закреплена на стене), шкива на втулке; 4 груза с зажимным винтом (табл. \ref{tab:m}), один подвесной груз.


\begin{table}[H]
	\caption{Вес грузов}
	\label{tab:m}
	\centering
	\pgfplotstableread[col sep = comma]{data/cargo.csv}{\loadedtable}
	\pgfplotstabletypeset{\loadedtable}
\end{table}

\section{Вывод формул} % (fold)
\label{sec:the}

\begin{figure}[H]
% \begin{center}
	\begin{minipage}[h]{0.49\linewidth}
	\centering
		\includestandalone[width=0.8\linewidth]{img/force_block}
		\subcaption{Шкив}	
	\end{minipage}
	\hfill 
	\begin{minipage}[h]{0.49\linewidth}
	\centering
		\includestandalone[width=0.8\linewidth]{img/force_cargo}
		\subcaption{Груз}	
	\end{minipage}
% \end{center}
\end{figure}

\subsection{Уравнения динамики} % (fold)

Запишем второй закон Ньютона для груза:
\begin{equation}
	m\vec{a}=m\vec{g}+\vec{T}
\end{equation}

Так как нить невесомая, то $T'=T$.

В проекции на $x$:
\begin{equation}
	\label{eq:ma}
	ma_x=mg-T_x
\end{equation}

Запишем уравнение моментов маятника в проекции на ось $z$ (ось от нас -- в стену):
\begin{equation}
	\label{eq:Ie}
	I\varepsilon_z=\sum_i M_{z(i)}=M_z^{T}+M_z^{R}
\end{equation}

Рассмотрим задачу в рамках одного периода колебаний (при следующем колебании физическая картина симметрична относительно отражения оси $z$)

Примем момент силы трения в оси шкива постоянным: $M_z^{R}=-M_0$

\subsubsection{Движение груза вниз} % (fold)

При движении нити справа (примем раскручивание: падение груза вниз) момент силы натяжения нити по определению $M_z^{T}=T_x\cdot r$

Так как нить нерастяжимая и движется без проскальзывания, то тангенциальное ускорение точек на шкиве и линейное груза будут равны, отсюда
\begin{equation}
	\label{eq:a-e}
	a_x=\varepsilon_z\cdot r,
\end{equation}   
где $r$ -- радиус шкива ($r=17.5\pm0.05$ мм).

Разрешим систему (\ref{eq:ma}), (\ref{eq:Ie}) относительно $\varepsilon_z$:
\begin{equation}
	I\varepsilon_z=M_z^{T}+M_z^{R}=T_x\cdot r -M_0\quad \Rightarrow \quad
	T_x=\frac{I \varepsilon + M_0}{r}
\end{equation}

Пользуясь (\ref{eq:a-e}), перепишем (\ref{eq:ma}):
\begin{equation}
	m\cdot \varepsilon_z r = mg - T_x = mg - \frac{I \varepsilon + M_0}{r}=\frac{mgr}{r}-\varepsilon\frac{I}{r}-\frac{M_0}{r}
\end{equation}
\begin{equation}
	\varepsilon_z(mr+\frac{I}{r})=\frac{mgr-M_0}{r}
\end{equation}
\begin{equation}
	\label{eq:e-t}
	\varepsilon_z=\frac{mgr-M_0}{mr^2+I}\\
\end{equation}

\subsubsection{Движение груза вверх} % (fold)

При движении нити слева (накручивание: подъем груза вверх) момент силы натяжения нити по определению $M_z^{T}=-T_x\cdot r$
\begin{equation}
	\label{eq:a-e2}
	a_x=-\varepsilon_z\cdot r,
\end{equation}   

Разрешим систему (\ref{eq:ma}), (\ref{eq:Ie}) относительно $\varepsilon_z$:
\begin{equation}
	I\varepsilon_z=M_z^{T}+M_z^{R}=-T_x\cdot r -M_0\quad \Rightarrow \quad
	T_x=-\frac{I \varepsilon + M_0}{r}
\end{equation}

Пользуясь (\ref{eq:a-e2}), перепишем (\ref{eq:ma}):
\begin{equation}
	m\cdot (-\varepsilon_z r) = mg - T_x = mg + \frac{I \varepsilon + M_0}{r}=\frac{mgr}{r}+\varepsilon\frac{I}{r}+\frac{M_0}{r}
\end{equation}
\begin{equation}
	\varepsilon_z(-mr-\frac{I}{r})=\frac{mgr+M_0}{r}
\end{equation}
\begin{equation}
	\label{eq:e+t}
	\varepsilon_z=-\frac{mgr+M_0}{mr^2+I}
\end{equation}



\subsection{Движение маятника} % (fold)

По определению, 
\begin{equation}
	\varepsilon_z=\frac{\mathrm d\omega_z}{\mathrm dt}
\end{equation}
\begin{equation}
	\omega_z=\frac{\mathrm d\phi}{\mathrm dt}
\end{equation}

Полагая начальные условия $\phi(t=0)=0, \vec\omega(t=0)=0$,
\begin{equation}
	\label{eq:omega-t}
	\omega_z(t)=\frac{mgr-M_0}{mr^2+I}\cdot t
\end{equation}
\begin{equation}
	\label{eq:phi-t}
	\phi(t)=\frac{mgr-M_0}{mr^2+I}\cdot \frac{t^2}{2}
\end{equation}

Обозначим время достижения грузом наинизшей точки $t_1$. Тогда
\begin{equation}
	\label{eq:omega1phi1}
	\omega_1=\omega_z(t_1), \quad \phi_1=\phi(t_1)
\end{equation}

\subsubsection{Рывок и потеря энергии} % (fold)

При достижении нижней точки происходит рывок. 

Если бы нить была идеальная, то полная механическая энергия груза бы не поменялась. В реальности часть энергии уходят на создание упругих волн в нити.

Таким образом, после рывка модуль скорости груза (а значит, и угловой скорости шкива) уменьшится (относительно скорости при движении вниз на этой же высоте).

Значит, \textit{при движении вверх} начальным условием будет не $\omega_z(t_1)=\omega_1$, а новое значение $\omega_z(t_1)=\omega_0$.

Угол, как непрерывная функция времени, своего значения не поменяет.

Тогда

\begin{equation}
	\label{eq:omega2}
	\omega_z(t)=\omega_0-\frac{mgr+M_0}{mr^2+I}\cdot (t-t_1)
\end{equation}
\begin{equation}
	\label{eq:phi2}
	\phi(t)=\phi_1+\omega_0(t-t_1)-\frac{mgr-M_0}{mr^2+I}\cdot \frac{(t-t_1)^2}{2}
\end{equation}

Разрешая систему (\ref{eq:omega2},\ref{eq:phi2},\ref{eq:omega1phi1}) относительно неизвестных величин, получаем
\begin{equation}
	I=\frac{mgr\left[t_2-t_1\right]^2}{\phi_2-\phi_1\left[1-(t_2/t_1-1)^2\right]}-mr^2
\end{equation}
\begin{equation}
	M_0=mgr\frac{\phi_2-\phi_1\left[1+(t_2/t_1-1)^2\right]}
				{\phi_2-\phi_1\left[1-(t_2/t_1-1)^2\right]}
\end{equation}
\begin{equation}
	\omega_0=2\frac{\phi_2-\phi_1}{t_2-t_1}, \qquad
	\omega_1=2\frac{\phi_1}{t_1}
\end{equation}

\subsubsection{Движение в отсутствие потерь энергии}

Следствием отсутствия потерь энергии будет два факта:

\begin{enumerate}
	\item Скорость не изменится после рывка, значит, $\omega_1=\omega_2$.
	\item Трения в оси не будет, значит, $M_0=0$.
\end{enumerate}

Тогда при движении \textit{вниз} будет:
\begin{equation}
	\omega_z(t)=\frac{mgr}{mr^2+I}\cdot t
\end{equation}
\begin{equation}
	\phi(t)=\frac{mgr}{mr^2+I}\cdot \frac{t^2}{2}
\end{equation}

Начальные условия при движении \textit{вверх} -- $\omega_1, \phi_1$:
\begin{equation}
	\label{eq:omega2}
	\omega_z(t)=\omega_1-\frac{mgr}{mr^2+I}\cdot (t-t_1)=\frac{mgr}{mr^2+I}\cdot (2t_1-t)
\end{equation}
\begin{gather}
	\label{eq:phi2}
	\phi(t)=\phi_1+\omega_1(t-t_1)-\frac{mgr}{mr^2+I}\cdot \frac{(t-t_1)^2}{2}=\\ \nonumber
	=\frac{mgr}{mr^2+I}\frac{t_1^2}{2}+\frac{mgr}{mr^2+I}(t_1t-t_1^2)-\frac{mgr}{mr^2+I}\cdot \frac{(t-t_1)^2}{2}=\\ \nonumber
	=\frac{mgr}{mr^2+I}\left[\frac{t_1^2}{2}-t_1^2+t_1t-\frac{t^2}{2}+t_1t-\frac{t_1^2}{2}\right]=\\ \nonumber
	=\frac{mgr}{mr^2+I}\left[
		2t_1t- \frac{(2t_1^2+t^2)}{2}
	\right]
\end{gather}
Легко убедиться, что подстановкой $t=t_1$ в уравнения для движения вверх получаются уравнения движения вниз, что соответствует допущению о отсутствии влияния рывка на скорость в нижней точке.


\begin{figure}[H]
\begin{center}
	\begin{minipage}[h]{0.49\linewidth}
	\centering
		\begin{tikzpicture}[scale=0.8]
			\begin{axis}[
					xlabel=$t$,
					ylabel=$\omega_z(t)$,
					xtick = {0},
					ytick = {0}
				]
				\addplot [domain=0:2]({x},x);
				\addplot [domain=2:4](x,{2*2-x});
			\end{axis}
		\end{tikzpicture}
	\end{minipage}
	\hfill 
	\begin{minipage}[h]{0.49\linewidth}
	\centering
		\begin{tikzpicture}[scale=0.8]
			\begin{axis}[
					xlabel=$t$,
					ylabel=$\phi(t)$,
					xtick = {0},
					ytick = {0}
				]
				\addplot [domain=0:2](x,{x^2/2});
				\addplot [domain=2:4](x,{4*x-4-x^2/2});
			\end{axis}
		\end{tikzpicture}
	\end{minipage}
\end{center}
\end{figure}

\subsubsection{Движение при моменте сил трения, пропорциональном угловой скорости}

Рассмотрим движение груза только вниз - т.е. размотку нити. Тогда можем воспользоваться ранее выведенной формулой (\ref{eq:e-t}):

\begin{equation*}
	\varepsilon(t)=\frac{mgr-M_0}{mr^2+I}
\end{equation*}

Примем момент сил трения пропорциональным угловой скорости:
\begin{equation}
	M_0=k\cdot\omega_z
\end{equation}
По определению
\begin{equation}
	\varepsilon(t)=\frac{d\omega_z}{dt}
\end{equation}
Тогда
\begin{equation}
	\frac{d\omega_z}{dt}=\frac{mgr-k\cdot\omega_z}{mr^2+I}
\end{equation}
Произведем разделение переменных:
\begin{equation}
	\frac{d\omega_z}{mgr-k\cdot\omega_z}=\frac{dt}{mr^2+I}
\end{equation}
Занесем функцию скорости под дифференциал:
\begin{equation}
	-\frac{1}{k}\frac{d(mgr-k\cdot\omega_z)}{mgr-k\cdot\omega_z}=\frac{dt}{mr^2+I}
\end{equation}
Интегрируем:
\begin{equation}
	-\frac{1}{k}\int_0^{\omega(t)}\frac{d(mgr-k\cdot\omega_z)}{mgr-k\cdot\omega_z}=\int_0^t\frac{dt}{mr^2+I}
\end{equation}
\begin{equation}
	-\frac{1}{k}[\ln(mgr-k\cdot\omega_z)-\ln(mgr-k\cdot0)]=
	\frac{t}{mr^2+I}
\end{equation}
\begin{equation}
	-\frac{1}{k}\ln\frac{mgr-k\cdot\omega_z}{mgr}=
	\frac{t}{mr^2+I}
\end{equation}
\begin{equation}
	\ln\frac{mgr-k\cdot\omega_z}{mgr}=
	\frac{-kt}{mr^2+I}
\end{equation}
Потенцируем:
\begin{equation}
	\frac{mgr-k\cdot\omega_z}{mgr}=
	e^{\frac{-kt}{mr^2+I}}
\end{equation}
\begin{equation}
	{mgr-k\cdot\omega_z}=
	{mgr}\cdot e^{\frac{-kt}{mr^2+I}}
\end{equation}
\begin{equation}
	k\cdot\omega_z=
	{mgr}(1-e^{\frac{-kt}{mr^2+I}})
\end{equation}
\begin{equation}
	\omega_z=
	\frac{mgr}{k}(1-e^{\frac{-kt}{mr^2+I}})
\end{equation}

\begin{figure}[H]
	\centering
	\begin{tikzpicture}[scale=1]
		\begin{axis}[
				xlabel=$t$,
				ylabel=$\omega_z(t)$,
				xtick = {0},
				ytick = {0},
				extra y ticks={1},
				extra y tick labels={$\frac{mgr}{k}$},
				samples = 1000
			]
			\addplot [domain=0:10](x,{1-e^(-x)});
			% \addplot [domain=2:4](x,{4*x-4-x^2/2});
			\addplot[color=black, dashed] coordinates {
			  (0,1) 
			  (10,1)
			};
		\end{axis}
	\end{tikzpicture}
\end{figure}

Угловая скорость блока  асимптотически возрастает. Максимальная скорость, к которой стремится $\omega_z$, легко найдется:
\begin{equation}
	\omega_z^{max}=\lim_{t\to+\infty} \frac{mgr}{k}(1-e^{\frac{-kt}{mr^2+I}})=\frac{mgr}{k}
\end{equation}

\subsection{Работа и энергия в системе маятник--груз}


\subsubsection{Потери кинетической энергии маятника при прохождении грузом нижней точки}

Запишем полную кинетическую энергию системы маятник--груз:
\begin{equation}
	W^\text{полн}=\frac{mv_\text{груза}^2}{2}+\frac{I\omega_\text{маятника}^2}{2}%+mgh_\text{груза}
\end{equation}

При прохождении грузом нижней точки необходимо рассмотреть два бесконечно близких момента времени.

В момент времени \textit{непосредственно \textbf{до} рывка} груз достигает нижней точки. Угловая скорость маятника в этот 	момент времени найдется по формуле (\ref{eq:omega-t}), а скорость груза -- нулевая.

Тогда
\begin{equation}
	W^\text{полн}_\text{до}=\frac{I\omega_1^2}{2}%+mgh_\text{груза}
\end{equation}

В момент времени \textit{непосредственно \textbf{после} рывка} высота груза не поменяется, скорость останется нулевой, изменится угловая скорость и примет значение $\omega_0$, которое мы можем найти экспериментально, и тогда

\begin{equation}
	W^\text{полн}_\text{после}=\frac{I\omega_0^2}{2}%+mgh_\text{груза}
\end{equation}

Таким образом, убыль кинетической энергии $-\Delta W$ составит
\begin{equation}
	-\Delta W=-(W^\text{полн}_\text{после}-W^\text{полн}_\text{до})=\frac{I}{2}(\omega_1^2-\omega_0^2)
\end{equation}

Как видно, изменение кинетической энергии произошло только у самого маятника.

Теперь можем найти, какую часть энергии маятник потерял при прохождении грузом нижней точки:

\begin{equation}
	\eta=\frac{-\Delta W}{W^\text{полн}_\text{до}}\times100\%=
	\frac{\omega_1^2-\omega_0^2}{\omega_1^2}\times100\%
\end{equation}

\subsubsection{Работа сил трения}

Запишем полную механическую энергию системы маятник--груз:
\begin{equation}
	W^\text{мех}=\frac{mv_\text{груза}^2}{2}+\frac{I\omega_\text{маятника}^2}{2}+mgh_\text{груза}
\end{equation}

Выберем ноль потенциальной энергии на $0$ оси $x$ (ось вниз, 0 -- точка пуска)

Рассмотрим случай, когда маятник совершает один период колебаний. Тогда полная механическая энергия в момент пуска будет составлять
\begin{equation}
	W^\text{мех}_\text{нач}=0,
\end{equation}

так как начальные условия - $\omega(t=0)=0, h(t=0)=0$.

В некий момент времени $t_2$ груз достигает верхней точки, и его полная механическая энергия составляет
\begin{equation}
	W^\text{мех}_\text{кон}=-mgh',
\end{equation}

Тогда изменение полной механической энергии
\begin{equation}
	W^\text{мех}_\text{нач}-W^\text{мех}_\text{кон}=mgh',
\end{equation}


Сила трения -- не единственная диссипативная сила в маятнике Обербека. При рывке в нижней точке часть энергии идет на создание упругих волн в нити, на деформацию подвеса, на создание поперечных колебаний груза.

Эту часть энергии мы нашли ранее: $\frac{I}{2}(\omega_1^2-\omega_0^2)$.

Тогда можем выразить:
\begin{equation}
	A_\text{тр}=W^\text{мех}_\text{нач}-W^\text{мех}_\text{кон}-\frac{I}{2}(\omega_1^2-\omega_0^2)=mgh'-\frac{I}{2}(\omega_1^2-\omega_0^2)
\end{equation}

где $h'$ -- расстояние от начала отсчета в верхней точке.

\subsection{Расчет зависимости момента инерции грузов}

Найдем зависимость момента инерции грузов на крестовине маятника в зависимости от их положения относительно оси вращения. 
\begin{figure}[H]
	\centering
	\includestandalone[width=0.5\linewidth]{img/I_cargo}
	\caption{Расстояние $r_1$ от оси вращения до центра масс груза}
	\label{fig:cargo-none}
\end{figure}

Примем известным момент инерции сплошного однородного цилиндра длины $l$, радиуса $r_2$ и массы $m$ относительно оси, перпендикулярной оси цилиндра и проходящей через центр масс (далее -- $q$-ось), параллельной оси вращения блока:
\begin{equation}
 	I_\text{ц}^\text{q}=f(l,r_2,m)={1 \over 4} m \cdot r_2^2 + {1 \over 12} m \cdot l^2
 \end{equation} 

По теореме Гюйгенса-Штейнера можем найти момент инерции груза относительно оси $z$:
\begin{gather}
	I_\text{гр}^z=I_\text{ц}^\text{q}(l,r_2,m_2)+m_2\cdot{r_1}^2=\\=
	{1 \over 4} m_2 \cdot r_2^2 + {1 \over 12} m_2 \cdot l^2+
	m_2\cdot r_1^2
\end{gather}

И для всех четырех грузов момент будет составлять:

\begin{gather}
	I_\text{4гр}^z=	m_2 \cdot r_2^2 + {1 \over 3} m_2 \cdot l^2+
	4m_2\cdot r_1^2
\end{gather}

\newpage
\section{Расчет погрешностей} % (fold)
\label{sec:ex}

\newpage
\section{Экспериментальные данные} % (fold)
\label{sec:ex}
\subsection{Компьютерная обработка данных}
    
Было проведено 4 серии опытов, в каждой по три опыта. 

Для снятия зависимости положения груза от времени была задействована видеосъемка.

Полученные видеофрагменты были обработаны программным пакетом \textbf{ffmpeg} для стабилизации изображения.

Далее c помощью ПО \textbf{Tracker} (is a free video analysis and modeling tool built on the Open Source Physics) в полуавтоматическом режиме была распознана на каждом кадре точка, служащая ориентиром для начала оси (в качестве такой точки была использована шляпка черного самореза над линейкой установки, т.к. высокая контрастность дает хороший \% распознования). 

Привязав к каждому кадру ось $x$, в полуавтоматическом режиме распознавалось положение груза на каждом кадре. Сформированный вывод сохранен в простейшем табличном формате \textbf{csv} (comma-separated values).

На основе экспериментальных данных в таблицы добавлены расчетные значения погрешностей и косвенных величин (работа с массивами данных производилась на математическом пакете \textbf{NumPy}, язык \textbf{Python}).

С помощью пакета \textbf{PgfPlots} на \textbf\LaTeX\ построены графики, соответствующие каждой серии опытов.  

\newpage
\subsection{Экспериментальное определение момента инерции}

\subsubsection{Маятник без дополнительных грузов}
\begin{figure}[H]
	\centering
	\includestandalone[width=0.8\linewidth]{img/cargo_none}
	\caption{Без грузов}
	\label{fig:cargo-none}
\end{figure}
\begin{gather}
	I_1=54201\pm 2761 \text{ g/cm}^2\\
	I_2=52789\pm 2112 \text{ g/cm}^2\\
	I_3=53366\pm 2292 \text{ g/cm}^2
	% \omega_0= 33.10 \text{c}^{-1}\\
	% \omega_1= 35.22 \text{c}^{-1}\\
	\eta= 11.64 \%\\
	A_\text{тр}= -2209594 \text{ erg}
\end{gather}

\subsubsection{Маятник с грузами около шкива}
\begin{figure}[H]
	\centering
	\includestandalone[width=0.8\linewidth]{img/cargo_start}
	\caption{Грузы зафиксированы у начала спиц}
	\label{fig:cargo-none}
\end{figure}
\begin{gather}
	I_1=68548\pm 2704 \text{ g/cm}^2\\
	I_2=63092\pm 2469 \text{ g/cm}^2\\
	I_3=62712\pm 2386 \text{ g/cm}^2\\
	\eta= 17.33 \%\\
	A_\text{тр}= -3391955 \text{ erg}
\end{gather}
Расчетный момент инерции грузов
\begin{equation}
	I=280\cdot(1.5)^2+1/3\cdot280\cdot(2.5)^2+4\cdot280\cdot(2.81)^2=10056\text{ g/cm}^2
\end{equation}

\subsubsection{Маятник с грузами в центре спиц}
\begin{figure}[H]
	\centering
	\includestandalone[width=0.8\linewidth]{img/cargo_middle}
	\caption{Грузы зафиксированы в центре спиц}
	\label{fig:cargo-none}
\end{figure}
\begin{gather}
	I_1=152823\pm 5195 \text{ g/cm}^2\\
	I_2=154595\pm 5223 \text{ g/cm}^2\\
	I_3=153261\pm 5291 \text{ g/cm}^2\\
	\eta= 15.63 \%\\
	A_\text{тр}= -3433231 \text{ erg}
\end{gather}
Расчетный момент инерции грузов
\begin{equation}
	I=280\cdot(1.5)^2+1/3\cdot280\cdot(2.5)^2+4\cdot280\cdot(9.75)^2=107683\text{ g/cm}^2
\end{equation}

\subsubsection{Маятник с грузами на концах спиц}
\begin{figure}[H]
	\centering
	\includestandalone[width=0.8\linewidth]{img/cargo_end}
	\caption{Грузы зафиксированы на концах спиц}
	\label{fig:cargo-none}
\end{figure}
\begin{gather}
	I_1=337206\pm 11102 \text{ g/cm}^2\\
	I_2=346030\pm 11667 \text{ g/cm}^2\\
	I_3=331027\pm 11421 \text{ g/cm}^2\\
	\eta= 15.29 \%\\
	A_\text{тр}= -3474774 \text{ erg}
\end{gather}
Расчетный момент инерции грузов
\begin{equation}
	I=280\cdot(1.5)^2+1/3\cdot280\cdot(2.5)^2+4\cdot280\cdot(17.5)^2=301213\text{ g/cm}^2
\end{equation}

\subsubsection{Зависимость момента инерции по удалению грузов от центра}
\begin{figure}[H]
	\centering
	\includestandalone[width=1\linewidth]{img/i_r}
	\caption{Зависимость $I(r)$, экспериментальная}
	\label{fig:cargo-none}
\end{figure}

Полученные экспериментально моменты инерции маятника, за вычетом начального момента инерции (без грузов), очень хорошо совпадают с расчетными значениями моментов инерции:

\begin{gather}
	63092 - 10056\approx 52789 \text{ g/cm}^2\\
	153261 - 107683\approx 52789 \text{ g/cm}^2\\
	353261 - 301213\approx 52789 \text{ g/cm}^2
\end{gather}

% \newpage
% \subsection{Затухание колебаний маятника}

% Дополнительно была снято затухание колебаний груза на маятнике.

% \begin{figure}[H]
% 	\centering	
% 	\includestandalone[width=0.8\linewidth]{img/cargo_osci}
% 	\caption{Затухающие колебания маятника без грузов}
% 	\label{fig:osci}
% \end{figure}

\subsection{Вращательная динамика блока}
\subsubsection{Зависимость угла поворота блока от времени}
\begin{figure}[H]
	\centering	
	\includestandalone[width=0.8\linewidth]{img/phi_t}
	\caption{Зависимость угла поворота блока от времени}
	\label{fig:osci}
\end{figure}

Зависимость угла поворота блока от времени снята для маятника без грузов.

Точка разворота произошла в момент времени $t=4.995\text{ c}$, груз прошел путь $S=150\text{ cm}$
Точка максимального подъема произошла в момент времени $t=8.692\text{ c}$, груз прошел путь $S=187\text{ cm}$
\subsubsection{Зависимость угловой скорости блока от времени}
\begin{figure}[H]
	\centering	
	\includestandalone[width=0.8\linewidth]{img/omega_t}
	\caption{Зависимость угловой скорости блока от времени}
	\label{fig:osci}
\end{figure}
Зависимость угловой скорости блока от времени снята для маятника без грузов.

В точке разворота груза маятником достигается максимальная угловая скорость $\omega_{max}=25.2\text{ c}^{-1}$.

\newpage
\section*{Заключение}
\addcontentsline{toc}{section}{Заключение}


\end{document}
