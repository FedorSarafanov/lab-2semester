\documentclass[tikz]{standalone}
\newcommand*{\source}{/home/lab/tex/templates}
\input{\source/head-img.tex}
\tikzset{%
  show curve controls/.style={
    postaction={
      decoration={
        show path construction,
        curveto code={
          \draw [blue] 
            (\tikzinputsegmentfirst) -- (\tikzinputsegmentsupporta)
            (\tikzinputsegmentlast) -- (\tikzinputsegmentsupportb);
          \fill [red, opacity=0.5] 
            (\tikzinputsegmentsupporta) circle [radius=.5ex]
            (\tikzinputsegmentsupportb) circle [radius=.5ex];
        }
      },
      decorate
}}}
\begin{document}
\tikzset{math3d/.style= {x={(1cm,0cm)}, z={(0,-0.4cm)}, y={(0cm,1cm)}}}
\begin{tikzpicture}[math3d]

\begin{scope}[canvas is xy plane at z=0]
	\draw[fill=cyan!40, draw=none] (-0.75,1) rectangle ++(1.5,2.5);
\end{scope}


\begin{scope}[canvas is xy plane at z=0]
	\draw[xshift=0.25cm]%[show curve controls]
  (0, 0) .. controls ++(90:0) and ++(20:-0.5) .. (0.4, 0.85) (0.5, 1) coordinate (1) 

  (-0.5, 0) .. controls ++(90:0) and ++(160:-0.5) .. (-0.9, 0.85) (-1, 1) coordinate (2);

  \fill[draw=none, fill=cyan!40]
  	(0.25,0) -- (0.65,0.85) -- (-0.65,0.85) -- (-0.25,0) -- cycle;

	\fill[xshift=0.25cm, draw=none, fill=white!40]%[show curve controls]
  (0, 0) .. controls ++(90:0) and ++(20:-0.5) .. (0.4, 0.85)  (-0.5, 0) .. controls ++(90:0) and ++(160:-0.5) .. (-0.9, 0.85); 

	\draw[xshift=0.25cm,  fill=white!40]%[show curve controls]
  (0, 0) .. controls ++(90:0) and ++(20:-0.5) .. (0.4, 0.85)  (-0.5, 0) .. controls ++(90:0) and ++(160:-0.5) .. (-0.9, 0.85); 

  \draw (1) -- ++ (0,4) -- ++(0.2,1);
  \draw (2) -- ++ (0,4) -- ++(-0.2,1);
  \fill[black!5] (1) ++ (0,4) -- ++(0.2,1) -- ++(-1.9,0) -- ++(0.2,-1) -- cycle;

\end{scope}
\begin{scope}[canvas is xz plane at y=0]
	\fill[draw=none, fill=cyan!40] (0,0) circle (0.25);
	\draw[] (0.25,0) arc (0:180:0.25);
	\draw[dashed] (0.25,0) arc (0:-180:0.25);
\end{scope}
\begin{scope}[canvas is xz plane at y=1]
	\draw[cyan!40, fill=cyan!40] (0,0) circle (0.75);
	\draw[] (0.75,0) arc (0:180:0.75);
	\draw[dashed] (0.75,0) arc (0:-180:0.75);
\end{scope}

\begin{scope}[canvas is xz plane at y=3.5]
	\draw[cyan!40, fill=cyan!40] (0,0) circle (0.75);
	\draw[] (0.75,0) arc (0:180:0.75);
	\draw[dashed] (0.75,0) arc (0:-180:0.75);
\end{scope}

\begin{scope}[canvas is xz plane at y=6]
	\draw[fill=black!5,] (0,0) circle (0.95);
	\draw[dashed, fill=white] (0,0) circle (0.15);


\end{scope}
\begin{scope}[canvas is xz plane at y=5]
	\draw[fill=black!5, draw=black!5] (0,0) circle (0.75);
	\draw[] (0.75,0) arc (0:180:0.75);
	\draw[dashed] (0.75,0) arc (0:-180:0.75);	
	\draw[dashed, fill=white] (0,0) circle (0.15);
\end{scope}
\draw[dashed] (-0.15,5,0) -- ++(0,1,0);
\draw[dashed] (0.15,5,0) -- ++(0,1,0);
\fill[white] (-0.15,5,0) rectangle ++(0.3,1,0);
\begin{scope}[canvas is xz plane at y=5]
	\draw[dashed, fill=white] (0,0) circle (0.15);
\end{scope}
\begin{scope}[canvas is xz plane at y=6]
	\draw[] (0,0) circle (0.95);
	\draw[dashed, fill=white] (0,0) circle (0.15);
\end{scope}

% строймастер домофоны
% 4300767
% 4308507

% \draw (0,0,0) -- (0,1,0);
% \begin{scope}[canvas is zy plane at x=0]

% 	\draw[dashed] (0,2) arc (90:270:2);
% 	\draw[] (0,-2) arc (-90:90:2);

% 	% \draw[dashed] (0,1) arc (90:270:1);
% 	% \draw[dashed] (0,-1) arc (-90:90:1);
% 	\draw[dashed] (0,0) circle (1);
% \end{scope}
% 	\draw (0,2,0) -- ++(4,0,0);
% 	\draw (0,-2,0) -- ++(4,0,0);
% 	\draw[dashed] (0,-1,0) -- ++(4,0,0);
% 	\draw[dashed] (0,1,0) -- ++(4,0,0);
% \begin{scope}[canvas is yz plane at x=4]

% 	\draw[] (0,0) circle (2);
% 	\draw[pattern=north east lines,pattern color=magenta!30] (0,0) circle (1);
% 	\draw (0,0) node[transform shape, rotate=90, scale=2, blue] {$S$};
% \end{scope}

\begin{scope}[canvas is xy plane at z=0]
	\begin{scope}
		\lineann[-1.5]{90}{3.5}{$h(t)$}
		% \lineann[1.5]{90}{-1}{$r$}
	\end{scope}
\end{scope}

% \begin{scope}[canvas is xy plane at z=0]
% 		\lineann[2]{0}{4}{$l$}
% \end{scope}

% Draw line annotation
% Input:
%   #1 Line offset (optional)
%   #2 Line angle
%   #3 Line length
%   #5 Line label
% Example:
%   \lineann[1]{30}{2}{$L_1$}

\end{tikzpicture}
\end{document}