\documentclass[tikz]{standalone}
\input{/home/lab/tex/templates/head.tex}

\tikzset{
  pics/carc/.style args={#1:#2:#3}{
    code={
      \draw[pic actions] (#1:#3) arc(#1:#2:#3);
    }
  },
  dash/.style={
  	dash pattern=on 5mm off 5mm
  }
}

\makeatletter

\tikzset{cheating dash to/.style args={on #1 off #2}{%
    to path={
        \pgfextra{
            \pgf@process{%
                % Scan \tikztostart
                \tikz@scan@one@point\pgfutil@firstofone(\tikztostart)%
                % Make correction if the node border point needs to be calculated
                \iftikz@shapeborder%
                    \pgf@process{\pgfpointshapeborder{\tikz@shapeborder@name}{\tikz@scan@one@point\pgfutil@firstofone(\tikztotarget)}}%
                \fi%
            }%
            \edef\tikztostart{\the\pgf@x,\the\pgf@y}%
            \pgf@xa=\pgf@x%
            \pgf@ya=\pgf@y%
            \pgf@process{%
                % Do the same for \tikztotarget
                \tikz@scan@one@point\pgfutil@firstofone(\tikztotarget)%
                \iftikz@shapeborder%
                    \pgf@process{\pgfpointshapeborder{\tikz@shapeborder@name}{\tikz@scan@one@point\pgfutil@firstofone(\tikztostart)}}%
                \fi%
            }%
            \edef\tikztotarget{\the\pgf@x,\the\pgf@y}%
            \advance\pgf@x by-\pgf@xa%
            \advance\pgf@y by-\pgf@ya%
            % \pgf@x and \pgf@y now contain the path vector
            \pgfmathveclen{\the\pgf@x}{\the\pgf@y}%
            % Same calculations as before
            \pgfmathparse{\pgfmathresult-#1}\let\rest=\pgfmathresult%
            \pgfmathparse{#1+#2}\let\onoff=\pgfmathresult%
            \pgfmathparse{max(floor(\rest/\onoff), 1)}\let\nfullonoff=\pgfmathresult%
            \pgfmathparse{max((\rest-\onoff*\nfullonoff)/\nfullonoff+#2, #2)}\let\offexpand=\pgfmathresult%
        }
            (\tikztostart) -- (\tikztotarget)
            \pgfextra{%
                % Have to do this here.
                \edef\tmp@dash{[dash pattern=on #1 off {\offexpand pt}]}%
                \expandafter\expandafter\expandafter\def\expandafter\expandafter\expandafter\tikz@after@path%
                    \expandafter\expandafter\expandafter{\expandafter\tikz@after@path\tmp@dash}%
            }%
        }       
    }
}

\makeatother

\begin{document}
\begin{tikzpicture}

		\xdef\R{10cm}

		\path (0,0) coordinate (1)  pic{carc=-30:-150:\R} coordinate (2); 

		\foreach \angle in {-30,-40,...,-90}{

			\pgfmathsetmacro\ANGLE{90+\angle}

			\draw (0,0) ++ (\angle:\R) -- ++ (\angle:0.4)
				node[below, scale=1.75, rotate={\angle+90}] {\pgfmathprintnumber[fixed, precision=0]{\ANGLE}};
		}

		\foreach \angle in {-90,-100,...,-150}{
			\pgfmathsetmacro\ANGLE{-(90+\angle)}

			\draw (0,0) ++ (\angle:\R) -- ++ (\angle:0.4)
				node[below, scale=1.75, rotate={\angle+90}] {\pgfmathprintnumber[fixed, precision=0]{\ANGLE}};
		}		

		\foreach \angle in {-35,-45,...,-145}{
			\draw (0,0) ++ (\angle:\R) -- ++ (\angle:0.2);
		}		


		% \draw[blue] (0,0) circle (1cm);
		% \foreach \angle in {0,10,...,358}{
		% 	\draw (0,0) -- ++ (\angle:1cm);
		% }				

		% \fill[black!50] (0,0) circle (0.5cm);


		% \draw[black!50, line width=0.2cm] (0,0.3cm) -- ++(0,1cm);
		% \draw[black!50, line width=0.2cm] (0,-0.3cm) -- ++(0,-1cm);

		% \draw[black!50, line width=0.2cm] (0,0) circle (1.3cm);

		% \fill[magenta] (0,-1.3cm) -- ++ (0.2cm,-0.3cm) -- ++ (-0.4cm,-0cm) -- cycle;

		% \fill[magenta, rotate=180] (0,-1.3cm) -- ++ (0.2cm,-0.3cm) -- ++ (-0.4cm,-0cm) -- cycle;



		% ВНЕШНЕЕ КОЛЬЦО
		\fill[black!20] (0,0) circle (4cm);

		\draw[black!60,line width=4mm] (-4cm,0) -- (-4.5cm,0);
		\draw[black!60,line width=4mm] (4cm,0) -- (4.5cm,0);		
			\begin{scope}
				\clip (2.5,-1) rectangle (4.1,1);
				\fill[black!40] (0,0) circle (4.1cm);
			\end{scope}
			\begin{scope}
				\clip (-2.5,-1) rectangle (-4.1,1);
				\fill[black!40] (0,0) circle (4.1cm);
			\end{scope}		

			\begin{scope}
				\clip (-0.5,2.5) rectangle (0.5,4.5);
				\fill[black!40] (0,0) circle (4.5cm);
			\end{scope}
			\begin{scope}
				\clip (-0.5,-2.5) rectangle (0.5,-4.5);
				\fill[black!40] (0,0) circle (4.5cm);
			\end{scope}					
		\fill[white] (0,0) circle (3cm);

		%ПОДШИПНИКИ ВНУТРЕННЕГО КОЛЬЦА
		\draw[dash,black!60,line width=4mm] (0,-1.5cm) -- (0,-3cm);
		\draw[dash,black!60,line width=4mm] (0,1.5cm) -- (0,3cm);

		% ВНУТРЕННЕЕ КОЛЬЦО
		\fill[black!20] (0,0) circle (3cm-5mm);
		\fill[black!30] (0,0) circle (2cm-5mm);

		% СТАТОР
		\fill[black!40] (0,0) circle (2cm-1.25cm);

		% ОСЬ Y
		\draw[dash pattern=on 2mm off 1mm,magenta!60,line width=2mm] (0,-3cm) -- (0,3cm);
		\draw[axis,->,dash pattern=on 2mm off 1mm,magenta!60,line width=2mm] (0,4.5cm) -- ++ (0,2) node[above, scale=2.5] {$+y$};

		% ОСь X
		\fill[white] (0,0) circle (0.5cm) node[magenta, scale=2] {$\bigodot$};
 		\contourlength{1mm};

		\node[right, scale=2.5, magenta, xshift=0.5em]  at (0,0) {\contour{white}{$+x$}};		

		% ОСЬ Z
		\draw[dash pattern=on 2mm off 1mm,magenta!60,line width=2mm] (-4.5cm,0) -- (-3cm,0);		
		\draw[dash pattern=on 2mm off 1mm,magenta!60,line width=2mm] (4.5cm,0) -- (3cm,0);

		\draw[axis,->,dash pattern=on 2mm off 1mm,magenta!60,line width=2mm] (-4.5cm,0) -- ++ (-2,0) node[left, scale=2.5] {$+z$};



		% \fill[magenta] (1.7cm,0) -- ++ (0.3cm,0.2cm) -- ++ (-0cm,-0.4cm) -- cycle;

		% \fill[magenta, rotate=180] (1.7cm,0) -- ++ (0.3cm,0.2cm) -- ++ (-0cm,-0.4cm) -- cycle;

		\draw (0,0) pic[->, red, line width=3pt, ]{carc=120:200:2cm};


		\begin{scope}
			\draw[] 	(-4.5,-1.5) 	-- 
						(-4.5,1.5) 		--
					++	(-2.5,0) 		--
					++	(-3.5,-2) 		--
					++	(0,-1)			--
					++	(0.25,0)		--
						(-10.25,-10.5)	--
					++ 	(3,-3)			--
						(0,-13.5)		;
			\draw[]		(-4.5,-1.5)		--
					++	(-1,0)			--
					++	(-3.16,-2)		--
					++	(0,-1.5);
			\draw[]		(-5.5,-1.5)		--
					++	(-5,0);			
		\end{scope}
		\begin{scope}[xscale=-1]
			\draw[] 	(-4.5,-1.5) 	-- 
						(-4.5,1.5) 		--
					++	(-2.5,0) 		--
					++	(-3.5,-2) 		--
					++	(0,-1)			--
					++	(0.25,0)		--
						(-10.25,-10.5)	--
					++ 	(3,-3)			--
						(0,-13.5)		;
			\draw[]		(-4.5,-1.5)		--
					++	(-1,0)			--
					++	(-3.16,-2)		--
					++	(0,-1.5);
			\draw[]		(-5.5,-1.5)		--
					++	(-5,0);			
		\end{scope}

		\draw (-7.25,-14) rectangle ++ (14.5,-1);
		\draw (-7.25,-13.5) rectangle ++ (14.5,-0.5);

		\draw (-7.25,-15) rectangle ++(2,-2);
		\draw[xscale=-1] (-7.25,-15) rectangle ++(2,-2);

\end{tikzpicture}
\end{document}