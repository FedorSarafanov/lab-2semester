\documentclass[a4paper,12pt]{article}%%%%%%%%%%%%
%%%%%%%%%%%%%%%%%%%%%%%%%%%%%%%%%%%%%%%%%%%%%%%%%
% Мне (да в прочем, и всем) свойственно идеализировать людей, и иногда не хочется, чтобы они были другими. А они другие, и все очень сложно получается)
\def\source{/home/lab/tex/templates}
%%%%%%%%%%%%%%%%%%%%%%%%%%%%%%%%%%%%%%%%%%%%%%%%%
\input{\source/head.tex}
%%%%%%%%%%%%%%%%%%%%%%%%%%%%%%%%%%%%%%%%%%%%%%%%%
\def\labauthor{Сарафанов Ф.Г.}
\def\labauthors{Сарафанов Ф.Г.}
\def\labnumber{16}
\def\labtheme{Определение вязкости воздуха}
%%%%%%%%%%%%%%%%%%%%%%%%%%%%%%%%%%%%%%%%%%%%%%%%%
%%%%%%%%%%%%%%%%%%%%%%%%%%%%%%%%%%%%%%%%%%%%%%%%%
\input{\source/math_left-of-equation.tex}
%%%%%%%%%%%%%%%%%%%%%%%%%%%%%%%%%%%%%%%%%%%%%%%%%
%%%%%%%%%%%%%%%%%%%%%%%%%%%%%%%%%%%%%%%%%%%%%%%%%

\geometry	
	{
		left			=	2cm,
		right 			=	2cm,
		top 			=	3cm,
		bottom 			=	3cm,
		bindingoffset	=	0cm
	}
\input{\source/lab_colontitle.tex}
% \usepackage{siunitx} % Formats the units and values
% \sisetup{
%   round-mode          = places, % Rounds numbers
%   round-precision     = 0, % to 2 places
% }
\pgfplotstableset{
    columns/mat/.style={%
    string type,
    column type=r,% 
    column name=\textsc{Груз}%
    },
    columns/m1/.style={column name=$m_1\text{, г}$},
    columns/m2/.style={column name=$m_2\text{, г}$},
    columns/m3/.style={column name=$m_3\text{, г}$},
    columns/m/.style={column name=$m\text{, г}$},
    columns/dr/.style={column name=$\Delta{d}\text{, мм}$},
    columns/d/.style={column name=$<d>$},
    columns/r/.style={column name=$<r>$},
    empty cells with={--}, % replace empty cells with ’--’
    every head row/.style={before row=\toprule,after row=\midrule},
    every last row/.style={after row=\bottomrule}
  }

\pgfplotsset{
    % most recent feature set of pgfplots
    compat=newest,
}

\newcommand\ct[1]{\text{\rmfamily\upshape #1}}

\newcommand*{\const}{\ct{const}}

\begin{document}

\input{\source/lab_title-page.tex} %Титульная страница

\tableofcontents
\newpage

\section*{Введение} % (fold)
\addcontentsline{toc}{section}{Введение}
\label{sec:input}

\section{Теория лабораторной работы}

\subsection{Эскиз установки} % (fold)
\label{sec:device}

\begin{figure}[H]
	\centering
	\includestandalone[width=0.45\linewidth]{img/c}
	% \caption{c}
	\label{fig:c}
\end{figure}

\subsection{Вывод формулы Пуазёйля}
    
\begin{figure}[H]
	\centering
	\includestandalone[width=0.8\linewidth]{img/pz1}
	\caption{pz1}
	\label{fig:pz1}
\end{figure}

Выделим из всей трубы радиусом $R$ цилиндр радиусом $r$. По формуле вязкого трения Ньютона, сила трения при движении такого цилиндра 
\begin{equation}
	f_R=\eta\frac{dv}{dr}S_\text{бок}
\end{equation}
С другой стороны, при перепаде давления на концах трубки $\Delta p$, будет действовать сила давления:
\begin{equation}
	f_p=\Delta p \cdot S
\end{equation}
Из условия стационарности потока скорость потока должна оставаться постоянной, т.е. сила давления уравновешивает силу трения:
\begin{equation}
	\eta\frac{dv}{dr}S_\text{бок}=\Delta p \cdot S
\end{equation}
\begin{equation}
	S_\text{бок}=2\pi r l
\end{equation}
\begin{equation}
	S=\pi r^2
\end{equation}
\begin{equation}
	\eta\frac{dv}{dr}2\pi r l=\Delta p \cdot \pi r^2
\end{equation}
Произведем разделение переменных:
\begin{equation}
	dv=\frac{\Delta p}{2\eta l}rdr
\end{equation}
Граничные условия $v(r=R)=0$, $v(r=0)\ne\infty$. Тогда расставим пределы и произведем интегрирование:
\begin{equation}
	\int\limits_0^{v(r)}dv=\int\limits_R^r\frac{\Delta p}{2\eta l}rdr
\end{equation}
Отсюда получаем параболическое распределение скорости в течении Пуазёйля: 
\begin{equation}
	v(r)=\frac{\Delta p (r^2-R^2)}{4\eta l}
\end{equation}
Расход жидкости через кольцевое сечение радиусом $r$, шириной $dr$ ($S=2\pi r dr$)
\begin{equation}
	dQ=S\cdot v=\frac{\Delta p (r^2-R^2)}{4\eta l}\pi r dr
\end{equation}
Тогда расход жидкости через все сечение трубы
\begin{equation}
	Q=\int dQ=\int\limits_0^R \frac{\Delta p (r^2-R^2)}{4\eta l}\pi r dr
\end{equation}
И окончательно получили формулу Пуазёйля:
\begin{equation}
	\label{Q}
	Q=\Delta p \frac{\pi R^4}{8\eta l}
\end{equation}


\newpage
\subsection{Расчет вязкости через экспериментальные данные}

С одной стороны, $Q$ нашли как функцию $Q=f(\Delta p, R, \eta, l)$ (см. \ref{Q}), с другой, объёмный расход можно выразить по определению:
\begin{equation}
	Q=-\frac{dV}{dt}=-S\frac{dh}{dt}
\end{equation}
Знак минус говорит об убыли жидкости.
Теперь можем записать дифференциальное уравнение:
\begin{equation}
	-S\frac{dh}{dt}=\Delta p \frac{\pi R^4}{8\eta l}
\end{equation}
Как было показано ранее, $\Delta p = \rho g h$. Тогда
\begin{equation}
 	S\frac{dh}{dt}=-\frac{\pi R^4}{8\eta l}\rho g h
 \end{equation} 
Обозначим 
\begin{equation}
	c=\frac{\pi R^4 \rho g}{8\eta lS}
\end{equation}
Тогда
\begin{equation}
	-\frac{dh}{dt}=ch
\end{equation}
\begin{equation}
	-\int\limits_{h_0}^{h(t)}\frac{dh}{h}=\int\limits_0^tcdt
\end{equation}
Отсюда получаем
\begin{equation}
	\ln\frac{h_0}h=ct
\end{equation}

Из предыдущего уравнения
\begin{equation}
	h(t)=h_0\cdot e^{-ct},\quad\text{где}\quad
	c=\frac{\pi R^4 \rho g}{8\eta lS}
\end{equation}

Таким образом, можно построением графика $\ln\frac{h_0}h$ от $t$ найти угловой коэффициент $c$, через который выразится вязкость воздуха:

\begin{equation}
	\eta=\frac{\pi R^4 \rho g}{8clS}
\end{equation}
где $\rho$ -- плотность жидкости, $R$ -- радиус капилляра, $l$ -- длина капилляра, $g$ -- ускорение свободного падения, $S$ -- площадь поперечного сечения сосуда, $c$ -- угловой коэффициент графика $\ln\frac{h_0}h$ от $t$.


\begin{equation}
	\label{ht}
	h(t)=h_0e^{-ct},\quad\text{где}\quad c=\frac{\pi r^4 \rho h}{8\eta lS}
\end{equation}

\subsection{Оценка уровня установления жидкости}

Выполнение лабораторной работы возможно при условии, что на концах капилляра установится квазистационарная разность давлений. 

Эта необходимость объясняется условием равенства расхода жидкости и воздуха.

Связав первоначальное понижение давления воздуха с расширением \textit{без} изменения количества, можем найти такое расстояние $b$, которое должен пройти уровень жидкости до установления квазистационарной	 разности давлений.

Воспользуемся законом Менделеева-Клапейрона на данных условиях:

\begin{equation}
	p_aV_0=p_aSl=\frac{m}{M} RT = \const
\end{equation}
Где $l$ -- высота воздушного столба в сосуде в начальный момент времени.

Если высота всего сосуда $H$, то
\begin{equation}
	l=H-h_0,
\end{equation}
где $h_0=h(t=0)$ -- высота водяного столба в сосуде в начальный момент времени.

Тогда высота водяного столба в момент установления давлений будет
\begin{equation}
	h=h_0-b
\end{equation}
Тогда
\begin{equation}
	p_al=(p_a-\rho g (h_0-b))(l+b)
\end{equation}
\begin{equation}
	p_al=p_al+
			p_ab-
			\rho g h_0l-
			\rho g h_0b+
			\rho g bl+
			\rho g b^2
\end{equation}
\begin{equation}
			\rho g b^2+
			b(p_a-\rho g h_0+\rho g l)
			-\rho g h_0l
			=
			0
\end{equation}
Тогда
\begin{equation}
	b=\frac{-(p_a-\rho g h_0+\rho g l)\pm\sqrt{(p_a-\rho g h_0+\rho g l)^2+4\rho^2g^2h_0l}}{2\rho g}
\end{equation}

Подставив параметры установки и лабораторные условия ($p_a=100258\text{ Па}$), получим пару решений.

Решением, имеющим физический смысл, будет
\begin{equation}
	b=1.5486\ldots
\end{equation}

Оценочно, погрешность вычисления $b>\Delta h_0=0.1 \text{ cm}$. Тогда необходимо <<откинуть>> незначащие разряды, и окончательно получим значение, на которое нужно дать опуститься уровню воды для установления разности давлений -- 
\begin{equation}
	b=1.5 \text{ cm}
\end{equation}

Однако, эта оценка получена при предположении, что за время установления воздух в сосуд не поступает.
В условиях эксперимента это не так. 

Понижение давления воздуха в сосуде будет замедлено постоянным увеличением объёма воздуха, а значит, до установления уровень жидкость пройдет большее расстояние.

Получили, что рассчитанное нами $b$ -- заниженная оценка. 

В эксперименте уровню воды дали опуститься на $9\text{ cm}$.


\subsection{Оценка ошибки определения давления}

Формула (\ref{ht}) зависимости $h(t)$ выведена в предположении о статичности перепада давлений на капилляре:
\begin{equation}
	\Delta p = \rho g h
\end{equation}
Этот закон достаточно хорошо совпадает с реальностью до начала колебаний жидкости. Однако из (\ref{ht}) можно выразить скорость жидкости:

\begin{equation}
	h(t)=h_0e^{-ct}
\end{equation}
\begin{equation}
	v=\frac{dh}{dt}=\frac{d[h_0e^{-ct}]}{dt}=-ch_0e^{-ct}
\end{equation}
Или, избавляясь от времени
\begin{equation}
	v=-ch
\end{equation}
Тогда запишем уравнение Бернулли для верхней и нижней точки жидкости в кране:
\begin{equation}
	p^*+\rho g h +\frac{\rho v^2}{2}=p_a+\frac{\rho v_k^2}{2}
\end{equation}
Где $v_k$ -- скорость жидкости в кране, которую можно выразить через расход жидкости в сечении сосуда и в сечении крана:
\begin{equation}
	Q_1=S_kv_k=Q_2=Sv \Rightarrow v_k = v\frac{S}{S_k}
\end{equation}
Откуда давление воздуха в сосуде $p^*$
\begin{equation}
	p^*=p_a-\rho g h-\frac{\rho v^2}{2}+\frac{\rho v_k^2}{2}
\end{equation}
То есть, в случае наличия динамического давления разность давлений будет
\begin{equation}
	\Delta p = p_a-p^*=\rho g h+\frac{\rho v^2}{2}-\frac{\rho v_k^2}{2}
\end{equation}
И наша ошибка в определении давления
\begin{equation}
	p_{err}=\frac{\rho v^2}{2}-\frac{\rho v_k^2}{2}=\frac12\rho c^2h^2[1-\frac{S^2}{S^2_k}]
\end{equation}

Т.е. максимальная ошибка, при подстановке экспериментальных значений составляет
\begin{equation}
	p_{err}=-5.821 \text{ Па},
\end{equation}
что составляет примерно $-0.1\%$ от значения $\rho g h$ на той же высоте.

\newpage
\section{Экспериментальные данные}
\subsection{Прямые измерения и график $\ln\frac{h_0}{h}$}

\begin{table}[H]
	\caption{Экспериментальные данные из протокола}
	\label{tab:m}
	\centering
	\pgfplotstableread[col sep = tab]{data/data.csv}{\loadedtable}
	\pgfplotstabletypeset[
		columns={h,tt1,tt2,tt3,t,pln},	
		% display columns/t/.style=
		% 	{
		% 		column name=$Value 2$,
		% 		column type={S},
		% 		string type
		% 	},
	columns/h/.style={column name=$h'$},			
	columns/tt1/.style={column name=$t_1$},			
	columns/tt2/.style={column name=$t_2$},			
	columns/tt3/.style={column name=$t_3$},			
	columns/t/.style={column name=$<t>$},			
	columns/pln/.style={column name=$\ln\frac{h_0}{h}$},			
	]{\loadedtable}
\end{table}

Из формулы (\ref{ht}) можно получить следующую: 
\begin{equation}
	ct=\ln\frac{h_0}{h}
\end{equation}
Которая есть ничто иное, как линейная зависимость. Построив график логарифма от времени, можно найти $c$ и выразить вязкость.
\begin{gather}
	f(h,h_0)=\ln\frac{h_0}{h}\\
	\Delta f=
		\pm\sqrt{
			\left(
				\frac{\partial f}{\partial h}\Delta h
			\right)^2+
			\left(
				\frac{\partial f}{\partial h_0}\Delta h
			\right)^2			
		}=
	\pm\Delta h\sqrt{\frac{1}{h^2}+\frac{1}{h_0^2}}
\end{gather}

\begin{figure}[H]
	\centering
	\includestandalone[width=1\linewidth]{img/ln}
	\caption{Зависимость $\ln\frac{h_0}h$ от времени}
	\label{fig:ln}
\end{figure}
Из графика, построенного по экспериментальным данным, нашли линейную регрессию графика $f(t)$. 

Так как после 400 секунды началось неравномерное вытекание воды, для поиска регрессии такие данные не учитывались.

Из графика нашли коэффициент $c$:
\begin{equation}
	c=2.19\cdot10^{-3}
\end{equation}

Максимальная относительная погрешность определения коэффициента 
\begin{equation}
	\varepsilon (c) = \varepsilon( f_{max}[t']) + \varepsilon (t')\approx 0.07
\end{equation}
И отсюда абсолютная
\begin{equation}
	\Delta c = 0.22\cdot10^{-3}
\end{equation}
Погрешность определения вязкости
\begin{equation}
	\varepsilon (\eta)=4\varepsilon(R)+\varepsilon (c)+\varepsilon (l)=
	4\frac{\Delta R}{R}+\frac{\Delta c}{c}=0.0004+0.07=0.0704
\end{equation}
И 
\begin{equation}
	\Delta \eta = 0.0704\cdot\eta=13.5 \text{ П}
\end{equation}

Подставляя опытные значения в следующую формулу, найдем вязкость воздуха:

\begin{equation}
	\eta=\frac{\pi R^4 \rho g}{8clS}=192[\pm13.5] \cdot 10^{-6} \text{ П}
\end{equation}


\subsection{Число Рейнольдса}

Число Рейнольдса характеризует переход от ламинарного течения к турбулентному. В нашем случае найдем $Re$ для течения в капилляре.
\begin{equation}
	Re_{кап}=\frac{\rho_\text{воздуха}\cdot v\cdot R_\text{капилляра}}{\eta}
\end{equation}
Скорость течения воздуха в капилляре пока неизвестна. Найдем её через условие равенства расходов воздуха и воды. С одной стороны,
\begin{equation}
	|Q|=\Delta p \frac{\pi R^4}{8\eta l}
\end{equation}
Но с другой стороны
\begin{equation}
	|Q|=S_\text{капилляра}v
\end{equation}
Отсюда
\begin{equation}
	v=\frac{\pi R^4}{8\eta l}\rho g h\frac{1}{S_\text{капилляра}}
\end{equation}
или
\begin{equation}
	v=ch\frac{S_\text{сосуда}}{S_\text{капилляра}}
\end{equation}
Тогда максимальная скорость будет
\begin{equation}
	v_{max}=2.19\cdot10^{-3}\cdot59\cdot\frac{18.5}{0.00062}=3299 \text{ см/с}
\end{equation}
И тогда
%0.001293
\begin{equation}
	Re_{max}=525 \leq 1100
\end{equation}

\newpage
\section*{Заключение}
\addcontentsline{toc}{section}{Заключение}



В проведенной лабораторной работе было найдено значение вязкости воздуха при температуре $t^\circ=25\ C$, давлении $p_a=100258 $ Па:
\begin{equation}
	\eta=192[\pm13.5] \cdot 10^{-6} \text{ П}
\end{equation}

Снята зависимость высоты уровня воды от времени, а также значение времени начала колебаний жидкости ($t^*=400 c$).

Оценено расстояние, на которое должен опуститься уровень воды для установления давления в верхней части сосуда $b$ (при условии постоянного объема воздуха):
\begin{equation}
	b=1.5\pm0.1 \text{ см}
\end{equation}

Показано, что данная оценка в реальности является заниженной.

Оценена относительна ошибка, вызванная предположением о статичности перепада давлений:
\begin{equation}
	p_{err}\approx -0.1\%\cdot \rho g h_0
\end{equation}

Найдено число Рейнольдса для течения воздуха в капилляре:
\begin{equation}
	Re_{max}=525 \leq 1100
\end{equation}
Откуда сделан вывод о ламинарности течения воздуха в капилляре.


\end{document}
